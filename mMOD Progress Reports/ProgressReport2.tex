\documentclass[letterpaper, 11pt]{article}
% \usepackage[utf8]{inputenc}
% \usepackage{sansmathfonts}
\usepackage[T1]{fontenc}
\usepackage[english]{babel}
\usepackage[margin=0.5in, head=40pt, top=1in, bottom=1in]{geometry} 
\usepackage{array}
\usepackage{booktabs}
\usepackage{microtype}
\usepackage{amsmath}
\usepackage{amsthm,amssymb,amsfonts, fancyhdr, color, comment, graphicx, environ, mathtools}
\usepackage[dvipsnames]{xcolor}
\usepackage{mdframed}
\usepackage[shortlabels]{enumitem}
\usepackage{parskip}
\usepackage{hyperref}
\usepackage{siunitx}
% \usepackage{ stmaryrd }
\usepackage{ textcomp }
\usepackage{import}
\usepackage{xifthen}
\usepackage{pdfpages}
\usepackage{transparent}
\usepackage{ar}
\usepackage{multirow}
\usepackage{lastpage}
\usepackage{longtable}
\usepackage{gensymb}
\usepackage[version=4]{mhchem}
\usepackage{caption}
\usepackage{tikz}
\usepackage{physics}
\usepackage{sectsty}
\usepackage{float}

\captionsetup{justification=centering,format=hang,labelsep=period,font={footnotesize}, labelfont={rm}}

% \renewcommand*\familydefault{\sfdefault} 

% \usepackage{bm}
\newcommand{\vect}[1]{\boldsymbol{\mathbf{#1}}}

\definecolor{scarlet}{HTML}{CC0033}
\hypersetup{
	colorlinks=true,
	linkcolor=blue,
	filecolor=magenta,      
	urlcolor=blue,
}

\newcommand{\incfig}[1]{%
	\def\svgwidth{\columnwidth}
	\import{./figures/}{#1.pdf_tex}
}
\graphicspath{ {./figures/} }
\pagestyle{fancy}

\global\mdfdefinestyle{equationbox}{%
	linecolor=black,linewidth=0.5pt,%
	backgroundcolor=white
}


\makeatletter
\let\oldabs\abs
\def\abs{\@ifstar{\oldabs}{\oldabs*}}
%
\let\oldnorm\norm
\def\norm{\@ifstar{\oldnorm}{\oldnorm*}}
\makeatother

\usepackage{lmodern}
% \usepackage[math]{iwona}
\usepackage{breqn}
\usepackage{mdframed}
\usepackage[subnum]{cases}
\usepackage{pdfpages}
\usepackage{soul}
\usepackage{nicematrix}
\usepackage[ruled, linesnumbered]{algorithm2e}
\lhead{Rutgers University mMOD - Anastasia Moreno}
\rhead{Inverse Isoparametric Mapping- \thepage \space of \pageref{LastPage}} 
\lfoot{}
\chead{}
\cfoot{Proprietary to the mMOD Group.}
% \usepackage[
%     backend=bibtex,
%     style=ieee,
% ]{biblatex}
% \addbibresource{./lib.bib}
\makeatletter
\DeclareRobustCommand{\vol}{\text{\volumedash}V}
\newcommand{\volumedash}{%
	\makebox[0pt][l]{%
		\ooalign{\hfil\hphantom{$\m@th V$}\hfil\cr\kern0.08em--\hfil\cr}%
	}%
}
\makeatother

\newcommand{\sgrav}{\ensuremath{\mathrm{SG}}}
\newcommand{\half}{\ensuremath{\frac{1}{2}}}
\newcommand{\tder}{\ensuremath{\frac{d}{dt}}}
\newcommand{\tsder}{\ensuremath{\frac{d^2}{dt^2}}}
\newcommand{\pde}[2]{\ensuremath{\frac{\partial#1}{\partial#2}}}
\newcommand{\ode}[2]{\ensuremath{\frac{d #1}{d #2}}}

\newcommand{\ihat}{\ensuremath{\vb{\hat{i}}}}
\newcommand{\jhat}{\ensuremath{\vb{\hat{j}}}}
\newcommand{\khat}{\ensuremath{\vb{\hat{k}}}}
\newcommand{\Ihat}{\ensuremath{\vb{\hat{I}}}}
\newcommand{\Jhat}{\ensuremath{\vb{\hat{J}}}}
\newcommand{\Khat}{\ensuremath{\vb{\hat{K}}}}
\newcommand{\qbarinf}{\ensuremath{\bar{q}_\infty}}
\newcommand{\qbar}{\ensuremath{\bar{q}}}
\newcommand{\rinf}{\ensuremath{\bar{\rho}_\infty}}
\newcommand{\meanchord}{\ensuremath{\bar{c}}}
\newcommand{\CMqc}{\ensuremath{C_{M_{c/4}}}}
\newcommand{\Cmqc}{\ensuremath{C_{m_{c/4}}}}
\newcommand{\CLa}{\ensuremath{C_{L_{\alpha}}}}
\newcommand{\Cla}{\ensuremath{C_{l_{\alpha}}}}
\newcommand{\CL}{\ensuremath{C_{L}}}
\newcommand{\Cl}{\ensuremath{C_{l}}}
\newcommand{\Mqc}{\ensuremath{M_{c/4}}}
\newcommand{\CD}{\ensuremath{C_{D}}}
\newcommand{\CDo}{\ensuremath{C_{D_0}}}
\newcommand{\CDI}{\ensuremath{C_{D,i}}}
\newcommand{\TSFC}{\ensuremath{\mathrm{TSFC}}}
\newcommand{\Vinf}{\ensuremath{V_{\infty}}}

\newcommand{\question}[1]{\colorbox{Cyan}{\texttt{Question #1}}}
\newcommand{\skewsym}[1]{\ensuremath{[\tilde{#1}]}}
\usepackage{matlab-prettifier}

\definecolor{backcolour}{rgb}{0.98,0.98,0.98}
\definecolor{darkgrey}{rgb}{0.1,0.1,0.1}
\lstdefinestyle{matlab}{
	backgroundcolor=\color{backcolour},   
	commentstyle=\color{LimeGreen}\itshape,
	keywordstyle=\color{RoyalBlue},
	numberstyle=\tiny\color{darkgrey},
	stringstyle=\color{Thistle},
	basicstyle=\ttfamily\footnotesize,
	breakatwhitespace=false,         
	breaklines=true,                 
	captionpos=b,                    
	keepspaces=true,                 
	numbers=left,                    
	numbersep=5pt,                  
	showspaces=false,                
	showstringspaces=false,
	showtabs=false,                  
	tabsize=2
}
\mdfsetup{%
	linecolor=black,
	linewidth=1pt,
	backgroundcolor=blue!5,
	roundcorner=0pt}
\lstset{style=matlab}



\begin{document}
	{
		\LARGE
		\textbf{Inverse Isoparametric Mapping of Trilinear Hexahedral Element}
		
	}
	\section{Introduction}
	This document explains the progress of the MATLAB implementation of an interactive GUI that maps a physical hexahedral element to its corresponding parent (or parametric) domain. The transformation utilizes the Newton–Raphson iterative scheme. The MATLAB code dynamically computes the parent coordinates $(\xi, \eta, \zeta)$ of a physical point selected via sliders and ensures that the mapping remains within valid boundaries.
	
	\section{Hexahedral Shape Functions and Isoparametric Mapping}
	A hexahedral element is defined by eight nodes, with standard trilinear shape functions given by:
	\begin{equation}
		N_i(\xi, \eta, \zeta) = \frac{1}{8} (1 + \xi \xi_i)(1 + \eta \eta_i)(1 + \zeta \zeta_i), \quad i = 1,2,\dots,8
	\end{equation}
	where $\xi_i, \eta_i, \zeta_i$ represent the coordinates of node $i$ in the parent domain.
	
	The forward mapping from parent to physical space is given by:
	\begin{equation}
		\mathbf{x} = \sum_{i=1}^{8} N_i(\xi, \eta, \zeta) \mathbf{x}_i,
	\end{equation}
	where $\mathbf{x}_i = (X_i, Y_i, Z_i)$ are the physical coordinates of node $i$.
	
	\section{Inverse Mapping using Newton-Raphson Iteration}
	Given a physical point $\mathbf{x} = (X,Y,Z)$, we seek the corresponding parent coordinates $(\xi, \eta, \zeta)$. This requires solving:
	\begin{equation}
		\mathbf{r}(\xi, \eta, \zeta) = \sum_{i=1}^{8} N_i(\xi, \eta, \zeta) \mathbf{x}_i - \mathbf{x} = 0.
	\end{equation}
	Using Newton–Raphson iteration:
	\begin{equation}
		\mathbf{J} \Delta \mathbf{p} = -\mathbf{r},
	\end{equation}
	where $\mathbf{J}$ is the Jacobian matrix:
	\begin{equation}
		\mathbf{J} = \begin{bmatrix} \frac{\partial x}{\partial \xi} & \frac{\partial y}{\partial \xi} & \frac{\partial z}{\partial \xi} \ \frac{\partial x}{\partial \eta} & \frac{\partial y}{\partial \eta} & \frac{\partial z}{\partial \eta} \ \frac{\partial x}{\partial \zeta} & \frac{\partial y}{\partial \zeta} & \frac{\partial z}{\partial \zeta} \end{bmatrix}.
	\end{equation}
	
	Each partial derivative in $\mathbf{J}$ is computed as:
	\begin{equation}
		\frac{\partial x}{\partial \xi} = \sum_{i=1}^{8} \frac{\partial N_i}{\partial \xi} X_i, \quad \text{similarly for } y \text{ and } z.
	\end{equation}
	
	The Newton-Raphson update is given by:
	\begin{equation}
		\begin{bmatrix} \xi^{(k+1)} \ \eta^{(k+1)} \ \zeta^{(k+1)} \end{bmatrix} = \begin{bmatrix} \xi^{(k)} \ \eta^{(k)} \ \zeta^{(k)} \end{bmatrix} - \mathbf{J}^{-1} \mathbf{r}.
	\end{equation}
	
	Iterations continue until $||\mathbf{r}||$ is below a specified tolerance.
	
	\section{Handling Boundary Clamping}
	Since the sliders are bounded by the axis-aligned bounding box, some selected points may lie outside the hexahedral boundary. If computed $(\xi, \eta, \zeta)$ values exceed $[-1,1]$, they are clamped, and the corrected physical coordinates are recomputed using the forward mapping:
	\begin{equation}
		\mathbf{x}{\text{new}} = \sum{i=1}^{8} N_i(\xi_{\text{clamped}}, \eta_{\text{clamped}}, \zeta_{\text{clamped}}) \mathbf{x}_i.
	\end{equation}
	This ensures the selected point remains inside the hexahedron.

	
	\end{document}